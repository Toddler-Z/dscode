\documentclass[UTF8]{ctexart}
\usepackage{geometry, CJKutf8}
\geometry{margin=1.5cm, vmargin={0pt,1cm}}
\setlength{\topmargin}{-1cm}
\setlength{\paperheight}{29.7cm}
\setlength{\textheight}{25.3cm}

% useful packages.
\usepackage{amsfonts}
\usepackage{amsmath}
\usepackage{amssymb}
\usepackage{amsthm}
\usepackage{enumerate}
\usepackage{graphicx}
\usepackage{subfigure}
\usepackage{multicol}
\usepackage{fancyhdr}
\usepackage{layout}
\usepackage{listings}
\usepackage{float, caption}

\lstset{
    basicstyle=\ttfamily, basewidth=0.5em
}

% some common command
\newcommand{\dif}{\mathrm{d}}
\newcommand{\avg}[1]{\left\langle #1 \right\rangle}
\newcommand{\difFrac}[2]{\frac{\dif #1}{\dif #2}}
\newcommand{\pdfFrac}[2]{\frac{\partial #1}{\partial #2}}
\newcommand{\OFL}{\mathrm{OFL}}
\newcommand{\UFL}{\mathrm{UFL}}
\newcommand{\fl}{\mathrm{fl}}
\newcommand{\op}{\odot}
\newcommand{\Eabs}{E_{\mathrm{abs}}}
\newcommand{\Erel}{E_{\mathrm{rel}}}

\begin{document}

\pagestyle{fancy}
\fancyhead{}
\lhead{张笑娟,3220101565}
\chead{数据结构与算法第7次作业}
\rhead{Dec.2nd, 2024}

\section{整体思路}
   对于vector,首先用$make\_heap$函数将其构建为一个最大堆。

   然后从堆的最后一个节点开始,依次向前,利用$pop\_heap$函数将堆顶最大元素弹出并且放到vector末尾,
   不过要注意的是弹出的堆顶元素依次从vector末尾往前放,且堆重排的时候不用考虑。

   弹出堆顶元素后,再利用percDown函数将除弹出元素以外的元素进行重排,具体做法是从堆顶开始,是父节点均比子节点大或者相等。
   具体实现是,将父节点与比其大的子节点进行调换,然后再调整子节点,直到不需要调整为止。

   就此,可以进行堆排序。

\section{测试流程}
首先,我写了产生随机、有序、逆序、重复序列的函数,然后写了check函数,用于检测排序是否正确。

再然后,我写了testheapsort和testsort\_heap的函数,内含check函数和chrono函数,分别用来测试heapsort和std::sort\_heap的性能以及运行时间和排序的正确性。

最后,我再main函数中生成随机、有序、逆序、重复序列,每个序列1000000个元素,分别测试heapsort和std::sort\_heap的性能。

\section{测试结果}   
由下表可见,heapsort和std::sort\_heap的性能差距不大,但std::sort\_heap的运行时间更短,每个用例少0.002s~0.003s左右。
\\
\begin{tabular}{|c|c|c|}
\hline
Sequence Type & heapsort Time (s) & std::sort\_heap Time (s) \\
\hline
Random & 0.099989 & 0.0964148 \\
Ordered & 0.0421415 & 0.0392575 \\
Reverse & 0.0477491 & 0.0453427 \\
Repetitive & 0.0515947 & 0.0488591 \\
\hline
\end{tabular}
    \\
测试时输出结果正常,无内存泄露。(AMD Ryzen 7 5800H)
\end{document}

%%% Local Variables: 
%%% mode: latex
%%% TeX-master: t
%%% End: 
