\documentclass[UTF8]{ctexart}
\usepackage{geometry, CJKutf8}
\geometry{margin=1.5cm, vmargin={0pt,1cm}}
\setlength{\topmargin}{-1cm}
\setlength{\paperheight}{29.7cm}
\setlength{\textheight}{25.3cm}

% useful packages.
\usepackage{amsfonts}
\usepackage{amsmath}
\usepackage{amssymb}
\usepackage{amsthm}
\usepackage{enumerate}
\usepackage{graphicx}
\usepackage{multicol}
\usepackage{fancyhdr}
\usepackage{layout}
\usepackage{listings}
\usepackage{float, caption}

\lstset{
    basicstyle=\ttfamily, basewidth=0.5em
}

% some common command
\newcommand{\dif}{\mathrm{d}}
\newcommand{\avg}[1]{\left\langle #1 \right\rangle}
\newcommand{\difFrac}[2]{\frac{\dif #1}{\dif #2}}
\newcommand{\pdfFrac}[2]{\frac{\partial #1}{\partial #2}}
\newcommand{\OFL}{\mathrm{OFL}}
\newcommand{\UFL}{\mathrm{UFL}}
\newcommand{\fl}{\mathrm{fl}}
\newcommand{\op}{\odot}
\newcommand{\Eabs}{E_{\mathrm{abs}}}
\newcommand{\Erel}{E_{\mathrm{rel}}}

\begin{document}

\pagestyle{fancy}
\fancyhead{}
\lhead{张笑娟,3220101565}
\chead{数据结构与算法第四次作业}
\rhead{Oct.21th, 2024}

\section{测试程序的设计思路}
\begin{enumerate}  
    \item 我在\texttt{List.h}文件中创建了\texttt{printList}函数,用来打印整个列表。  
    \item 我利用默认构造函数初始化了一个列表\texttt{lst1},并调用\texttt{printList}函数打印了它。  
    \item 在\texttt{lst1}中,我利用for循环函数和\texttt{push\_back}函数向\texttt{lst1}中插入了整数0到9,并且用\texttt{push\_front}函数插入了整数5。接着调用\texttt{printList}函数打印了\texttt{lst1}。  
    \item 我利用初始化列表构造函数,拷贝构造函数,移动构造函数,赋值运算符依次创造了\texttt{lst2}, \texttt{lst3}, \texttt{lst4}, \texttt{lst5}四个列表,并调用\texttt{printList}函数打印了它们。其中,\texttt{lst2}初始化了1到5的整数列表,\texttt{lst2}拷贝到\texttt{lst3},\texttt{lst3}移动到\texttt{lst4},\texttt{lst1}赋值给\texttt{lst5}。  
    \item 在\texttt{lst2}中,我利用\texttt{begin()}函数和++运算符,将\texttt{begin}变量指向第二个元素,并利用\texttt{insert()}函数,在这个位置插入6;同时,我利用\texttt{end()}函数和\texttt{--}运算符,将\texttt{end}变量指向倒数第二个元素,并利用\texttt{erase}函数,删去这个位置的元素。接着调用\texttt{printList}函数打印了\texttt{lst2}。  
    \item 在\texttt{lst5}中,我利用\texttt{pop\_front}函数和\texttt{pop\_back}函数,分别删除了第一个和最后一个元素,之后我调用\texttt{front()}函数和\texttt{back()}函数,打印了删除后的第一个和最后一个元素。
    然后,我利用size函数,打印了列表的大小。最后,我打印了\texttt{lst5}。
    \item 在\texttt{lst4}中,我利用\texttt{begin()}函数和++运算符,将\texttt{it1}指向第三个元素;利用\texttt{end()}函数和\texttt{--}运算符,将\texttt{it2}指向倒数第一个元素。接着,我利用\texttt{erase(it1, it2)}函数,删去了第三个元素到第五个元素(不包括第五个元素)之间的元素。
    接着我调用\texttt{printList}函数打印了\texttt{lst4}。然后我利用\texttt{empty}函数,判断了\texttt{lst4}是否为空。最后,我利用\texttt{clear}函数,清空了\texttt{lst4},并且再次用\texttt{empty}函数判断了\texttt{lst4}是否为空。

\end{enumerate}

\section{测试的结果}  
 
    测试结果一切正常。  
      
    我用 valgrind 进行测试,发现没有发生内存泄漏。  
      
\section{(可选)bug报告}  
      
    我发现了一个 bug,触发条件如下:  
\begin{enumerate}  
    \item 首先,我在第6步中,不使用\texttt{lst5}进行测试,而是使用\texttt{lst3}进行测试。  
    \item 然后,程序运行到第5步结束就会终止,后续不再输出。  
    \item 经过反复确认,是\texttt{pop\_front}函数,\texttt{pop\_back}函数,\texttt{front}函数以及\texttt{back}函数无法正确工作。  
\end{enumerate}  
      
    据我分析,它出现的原因是:在第4步中,我将\texttt{lst3}移动到\texttt{lst4},但是由于移动构造函数的原因,\texttt{lst3}的元素并没有被拷贝,而是直接移动到了\texttt{lst4}。此时,\texttt{lst3}处于有效但为空的状态,导致\texttt{pop\_front}函数,\texttt{pop\_back}函数,\texttt{front}函数以及\texttt{back}函数无法正确工作。\\
   
    改正措施:利用\texttt{lst5}进行测试,而不是\texttt{lst3}。     

\end{document}

%%% Local Variables: 
%%% mode: latex
%%% TeX-master: t
%%% End: 
