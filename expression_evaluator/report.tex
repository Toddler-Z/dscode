\documentclass[UTF8]{ctexart}
\usepackage{geometry, CJKutf8}
\geometry{margin=1.5cm, vmargin={0pt,1cm}}
\setlength{\topmargin}{-1cm}
\setlength{\paperheight}{29.7cm}
\setlength{\textheight}{25.3cm}

% useful packages.
\usepackage{amsfonts}
\usepackage{amsmath}
\usepackage{amssymb}
\usepackage{amsthm}
\usepackage{enumerate}
\usepackage{graphicx}
\usepackage{subfigure}
\usepackage{multicol}
\usepackage{fancyhdr}
\usepackage{layout}
\usepackage{listings}
\usepackage{float, caption}

\lstset{
    basicstyle=\ttfamily, basewidth=0.5em
}

% some common command
\newcommand{\dif}{\mathrm{d}}
\newcommand{\avg}[1]{\left\langle #1 \right\rangle}
\newcommand{\difFrac}[2]{\frac{\dif #1}{\dif #2}}
\newcommand{\pdfFrac}[2]{\frac{\partial #1}{\partial #2}}
\newcommand{\OFL}{\mathrm{OFL}}
\newcommand{\UFL}{\mathrm{UFL}}
\newcommand{\fl}{\mathrm{fl}}
\newcommand{\op}{\odot}
\newcommand{\Eabs}{E_{\mathrm{abs}}}
\newcommand{\Erel}{E_{\mathrm{rel}}}

\begin{document}

\pagestyle{fancy}
\fancyhead{}
\lhead{张笑娟,3220101565}
\chead{四则运算求值程序设计报告}
\rhead{Dec.17, 2024}


\section{项目背景}
本项目实现了一个能处理中缀表达式的s四则运算求值程序,包括加法、减法、乘法、除法和括号,可以处理负数和科学计数法(如-2.3e-2.4)。程序还能检测非法输入。

\section{设计思路}
\subsection{算法概要}
程序通过中缀转后缀表达式的方法实现求值。首先做一些前期准备,包括判断数字字符、判断运算符、计算运算符的优先级、判断括号匹配、编写后缀运算函数。
然后在evaluate函数中,进行后缀表达式的求值。具体步骤如下:
\begin{itemize}
    \item 判断括号是否匹配;
    \item 定义两个栈,一个存储运算数(operands),一个存储运算符(operators);
    \item 对表达式进行逐个遍历;
    \item 如果遇到空格,跳过;
    \item 如果遇到数字字符或者负号(位于字符串的开头或位于运算符之后或位于左括号之后),则将下一个运算符之前的所有数字字符转化为一个数字,并将数字压入operands栈,具体步骤如下:
       
        \begin{enumerate}
            \item 定义一个double变量operand初始值为0;
            \item 定义一个bool变量isNegative初始值为false,如果遇到负号,则设置为true,并跳过负号;
            \item 处理整数部分:从当前位置开始,将其转化为数字,并将数字加上operand乘以10,并将结果赋值给operand,然后下标加1,
            直到遇到非数字字符或表达式结束,跳出循环;
            \item 处理小数部分:如果遇到小数点,先跳过小数点,再将下一个字符转化为数字,并计算小数位数j,并将数字乘以10的负j次方,并将结果加上operand,然后赋值给operand,然后下标加1,
            直到遇到非数字字符或表达式结束,跳出循环;
            \item 处理指数部分:如果遇到字符e,先跳过字符e,再将下一个运算符或结尾前的所有数字字符转化为一个数字exponent,并将operand乘以10的exponent次方,赋值给operand;
            \item 如果isNegative为true,则将operand乘以-1;
            \item 将operand压入operands栈;
        \end{enumerate}

    \item 如果遇到运算符,如果operators栈不为空,且当前运算符优先级小于或等于operators栈顶运算符的优先级,则先处理operators栈顶的运算符的运算,然后则将当前运算符压入operators栈。否则直接将当前运算符压入operators栈;
    \item 如果遇到左括号,则直接压入operators栈;
    \item 如果遇到右括号,则处理operators栈顶的运算符的运算,直到遇到左括号为止,然后弹出左括号;
    \item 当表达式遍历完成后,处理operators栈中剩余的运算符的运算,并将运算结果压入operands栈;
    \item 从operands栈中弹出最后一个结果,作为表达式的最终结果。

\end{itemize}

\subsection{合法性检查}
- 检查括号匹配。 

- 检测运算符在表达式开头或结尾、连续运算符、除数为零等错误。  

\section{测试思路及结果}
测试思路如下:
\begin{enumerate}
    \item 初始化表达式为: 。
    \item 输入合法表达式,则输出表达式及其求值结果。若输入非法,则输出ILLEGAL。若输入exit,则退出程序。
\end{enumerate}

测试结果如下表所示:

\bigskip

\begin{tabular}{|l|l|l|}
\hline
\textbf{输入表达式} & \textbf{结果} & \textbf{备注}    \\ \hline
3+4*(5-2*2)/2        & 3+4*(5-2*2)/2 = 5          & 基本运算      \\ \hline
-3*-3+2/-2--1+-5                & -3*-3+2/-2--1+-5 = 4       & 负数      \\ \hline
-2.2e2.2--2.2e-0.5               & -2.2e2.2--2.2e-0.5 = -347.981           & 科学计数法    \\ \hline
3-(2*(4-5*(1-0.4/2))-3)        & 3-(2*(4-5*(1-0.4/2))-3) = 6  & 多层括号    \\  \hline
3-(2*(4-5*(1-0.4/2))-3        & ILLEGAL: Parentheses are not balanced   & 括号不匹配   \\ \hline
*3-4   &ILLEGAL: Invalid operator & 运算符在开头 \\ \hline
3-+4   &ILLEGAL: Invalid operator   & 运算符连续 \\ \hline
3+4-   &ILLEGAL: Invalid operator   & 运算符在结尾 \\ \hline
4 /(3-3)  &ILLEGAL: Division by zero   & 除数为零 \\ \hline
exit &    & 退出程序 \\ \hline
\end{tabular}
\section{总结}
本项目实现了一个功能完善的四则运算求值程序,支持负数和科学计数法,并能检测非法输入。程序无内存泄露。

\end{document}