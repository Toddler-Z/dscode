\documentclass[UTF8]{ctexart}
\usepackage{geometry, CJKutf8}
\geometry{margin=1.5cm, vmargin={0pt,1cm}}
\setlength{\topmargin}{-1cm}
\setlength{\paperheight}{29.7cm}
\setlength{\textheight}{25.3cm}

% useful packages.
\usepackage{amsfonts}
\usepackage{amsmath}
\usepackage{amssymb}
\usepackage{amsthm}
\usepackage{enumerate}
\usepackage{graphicx}
\usepackage{subfigure}
\usepackage{multicol}
\usepackage{fancyhdr}
\usepackage{layout}
\usepackage{listings}
\usepackage{float, caption}

\lstset{
    basicstyle=\ttfamily, basewidth=0.5em
}

% some common command
\newcommand{\dif}{\mathrm{d}}
\newcommand{\avg}[1]{\left\langle #1 \right\rangle}
\newcommand{\difFrac}[2]{\frac{\dif #1}{\dif #2}}
\newcommand{\pdfFrac}[2]{\frac{\partial #1}{\partial #2}}
\newcommand{\OFL}{\mathrm{OFL}}
\newcommand{\UFL}{\mathrm{UFL}}
\newcommand{\fl}{\mathrm{fl}}
\newcommand{\op}{\odot}
\newcommand{\Eabs}{E_{\mathrm{abs}}}
\newcommand{\Erel}{E_{\mathrm{rel}}}

\begin{document}

\pagestyle{fancy}
\fancyhead{}
\lhead{张笑娟,3220101565}
\chead{数据结构与算法第六次作业}
\rhead{Nov.11th, 2024}

\section{remove函数的阐述}
    课本上不得不使用递归的原因在于findMin函数功能的缺失,findMin函数只支持返回最小节点的指针。minNode指针只是作为一个辅助指针,并不能真正指向最小节点,也不能对节点进行操作。
    而remove函数需要删除指定节点,因此在执行删除操作时不可避免地要进行递归操作,找到最小节点并删除,没有效率。
     
    所以,我们可以再定义一个detachMin函数,使其既可以找到最小节点,又可以执行删除操作。然后再使用remove函数,调用detachMin函数找到最小节点并删除,同时返回指向最小节点的指针。
    \begin{figure}[htbp] % [htbp]表示浮动选项,可以放置图片在合适位置
        \centering
        \includegraphics[width=0.8\textwidth]{remove.png} % 插入图片,设置宽度为页面宽度的80%
        \caption{remove函数} % 图片标题
        \label{1} % 给图片添加标签,便于引用
    \end{figure}
    
    上一次作业我认为t=minNode是节点的复制,而t->element = minNode->element是节点内容的复制,不是节点的复制,我理解错了。

    t=minNode其实只是个链接,因此在这次作业中,只需要将t节点的左子树和右子树链接到minNode的左子树和右子树位置,然后将t节点位置链接到minNode节点即可。同时,为了避免内存泄露,需要将原来t节点指向的空间销毁。

\section{测试的结果}   
 
    测试时输出结果正常,无内存泄露,开启 O2 编译优化后实际用时为0.801s(AMD Ryzen 7 5800H)。
\end{document}

%%% Local Variables: 
%%% mode: latex
%%% TeX-master: t
%%% End: 